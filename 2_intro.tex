\chapter{Введение}
\label{ch:intro}

\section{Актуальность проблемы}
Ассиметричное шифрование (шифрование с открытым ключом) лежит в основе множества ключевых сетевых протоколов (TLS, SSH, HTTPS). Однако устойчивость используемых в данный момент алгоритмов шифрования находится под угрозой квантовых компьютеров. При создании достаточно большого (в терминах количества кубитов -- квантовых битов) компьютера, уже разработанные квантовые алгоритмы позволят получить доступ к огромному массиву информации, передающейся по сети. Под угрозой находятся переписки, банковские транзакции, личные данные, системы удаленного управления и т.п.

Для защиты от атак с помощью квантовых компьютеров разрабатываются специальные -- постквантовые алгоритмы. Основным отличием от классических алгоритмов они отличаются требованием устойчивости как к классическим, так и квантовым атакам. При этом шифрование должно производится на классических компьютерах. Это позволит внедрить алгоритмы заранее и использовать повсеместно.

Стратегия "Harvest now, decrypt later"\; заключается в сборе открытых ключей и зашифрованных данных сейчас, с расчетом на появление технологий, позволяющих расшифровать их в будущем. Это позволит злоумышленникам воспользоваться слабостью нынешнего шифрования. Взлом этих алгоритмов также позволит подменить подписи, использованные в прошлом, переписав историю транзакций (например, в блокчейне). Поэтому разработка и внедрение постквантовых алгоритмов должно производится задолго до того, как появятся квантовые компьютеры, представляющие угрозу (Y2Q или Q-day).

\section{Механизмы шифрования ключа (KEM)}

Задачи создания защищенного канала решаются наиболее быстро и удобно с помощью симметричного шифрования. В этом случае обе стороны используют общий секретный ключ. Недостаток этого подхода заключается в необходимости доставить секретный ключ, не раскрыв его злоумышленникам. В случае обеспечения такого доступа между агентами по сети этот недостаток становится критическим. Для его исправления используются KEM (Key Encapsulation Mechanism -- механизм шифрования ключа).

KEM основаны на ассиметричном шифровании с открытым ключом. Агент A генерирует асимметричную пару ключей $(s, p)$, открытый ключ $p$ отправляет по сети агенту B. Агент B генерирует симметричный секретный ключ $k$, шифрует его с помощью открытого ключа $p$ и отправляет шифр $c$. Агент A дешифрует $c$ с помощью $s$, получая симметричный ключ $k$.

\section{Классические алгоритмы}

Все алгоритмы с открытым ключом опираются на сложность инвертирования некоторой односторонней функции. Такая функция должна иметь полиномиальную сложность при вычислении и экспоненциальную сложность при инвертировании. В прикладном смысле это означает, что шифрование с помощью открытого ключа должно быть быстрым, а дешифровка быстрой только с помощью секретного ключа.

На данный момент самые широкораспространенные алгоритмы используют факторизацию целых чисел (например, RSA), дискретный логарифм (EdDSA). Все эти задачи уязвимы к алгоритму Шора -- квантовому алгоритму, который использует квантовое преобразование Фурье (QFT).

\section{Постквантовые алгоритмы}

Постквантовые алгоритмы основаны на задачах из различных областей математики, включая:
\begin{itemize}
	\item Решетки (Lattice-based);
	\item Хэши (Hash-based);
	\item Коды ошибок (Code-based);
	\item Изогении (Isogeny-based).
\end{itemize}

Мы рассмотрим принципы работы нескольких из них.

\endinput