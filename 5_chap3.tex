\chapter{Заключение}

Приход квантовых компьютеров угрожает большому количеству используемых алгоритмов, поэтому конфиденциальность информации фундаментально зависит от развития постквантового шифрования .
Более того, оно стимулирует развитие криптографии в целом, повышая актуальность исследования как квантовых, так и неквантовых алгоритмов шифрования и взлома.

К счастью, этой области уделяется все больше внимания как со стороны государственных структур \cite{nsa_pqc}, так и частных компаний \cite{nvidia_pqc}. Например, NIST проводит отбор и стандартизацию алгоритмов постквантового шифрования. В процессе отбора были найдены важные уязвимости, в том числе в последнем раунде был взломан вышеописанный алгоритм SIDH, причем неквантовым алгоритмом \cite{sidh_broken}. Более тщательное изучение уже используемых алгоритмов (например, McEliece) повышает уверенность в том, что они действительно надежны в том числе против классических атак.

Задачи, лежащие в основе шифрования, опираются на новые, сложные абстракции. Использование этих задач в шифровании, поиск уязвимостей в алгоритмах приводит к более глубоким математическим исследованиям, развивая математическую область, на которую опирается шифрование.

Таким образом, потенциальное появление квантовых компьютеров повысило внимание, уделяемое безопасности информации, а оно, в свою очередь, дало толчок к развитию математических и криптографических направлений.

\endinput